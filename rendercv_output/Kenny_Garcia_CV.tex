\documentclass[10pt, letterpaper]{article}

% Packages:
\usepackage[
    ignoreheadfoot, % set margins without considering header and footer
    top=0.5 cm, % seperation between body and page edge from the top
    bottom=0 cm, % seperation between body and page edge from the bottom
    left=2 cm, % seperation between body and page edge from the left
    right=2 cm, % seperation between body and page edge from the right
    footskip=0.0 cm, % seperation between body and footer
    % showframe % for debugging 
]{geometry} % for adjusting page geometry
\usepackage{titlesec} % for customizing section titles
\usepackage{tabularx} % for making tables with fixed width columns
\usepackage{array} % tabularx requires this
\usepackage[dvipsnames]{xcolor} % for coloring text
\definecolor{primaryColor}{RGB}{0, 0, 0} % define primary color
\usepackage{enumitem} % for customizing lists
\usepackage{fontawesome5} % for using icons
\usepackage{amsmath} % for math
\usepackage[
    pdftitle={Kenny Garcia's CV},
    pdfauthor={Kenny Garcia},
    pdfcreator={LaTeX with RenderCV},
    colorlinks=true,
    urlcolor=primaryColor
]{hyperref} % for links, metadata and bookmarks
\usepackage[pscoord]{eso-pic} % for floating text on the page
\usepackage{calc} % for calculating lengths
\usepackage{bookmark} % for bookmarks
\usepackage{lastpage} % for getting the total number of pages
\usepackage{changepage} % for one column entries (adjustwidth environment)
\usepackage{paracol} % for two and three column entries
\usepackage{ifthen} % for conditional statements
\usepackage{needspace} % for avoiding page brake right after the section title
\usepackage{iftex} % check if engine is pdflatex, xetex or luatex

% Ensure that generate pdf is machine readable/ATS parsable:
\ifPDFTeX
    \input{glyphtounicode}
    \pdfgentounicode=1
    \usepackage[T1]{fontenc}
    \usepackage[utf8]{inputenc}
    \usepackage{lmodern}
\fi

\usepackage{charter}

% Some settings:
\raggedright
\AtBeginEnvironment{adjustwidth}{\partopsep0pt} % remove space before adjustwidth environment
\pagestyle{empty} % no header or footer
\setcounter{secnumdepth}{0} % no section numbering
\setlength{\parindent}{0pt} % no indentation
\setlength{\topskip}{0pt} % no top skip
\setlength{\columnsep}{0.15cm} % set column seperation
\pagenumbering{gobble} % no page numbering

\titleformat{\section}{\needspace{4\baselineskip}\bfseries\large}{}{0pt}{}[\vspace{1pt}\titlerule]

\titlespacing{\section}{
    % left space:
    -1pt
}{
    % top space:
    0.3 cm
}{
    % bottom space:
    0.2 cm
} % section title spacing

\renewcommand\labelitemi{$\vcenter{\hbox{\small$\bullet$}}$} % custom bullet points
\newenvironment{highlights}{
    \begin{itemize}[
        topsep=0.10 cm,
        parsep=0.10 cm,
        partopsep=0pt,
        itemsep=0pt,
        leftmargin=0 cm + 10pt
    ]
}{
    \end{itemize}
} % new environment for highlights


\newenvironment{highlightsforbulletentries}{
    \begin{itemize}[
        topsep=0.10 cm,
        parsep=0.10 cm,
        partopsep=0pt,
        itemsep=0pt,
        leftmargin=10pt
    ]
}{
    \end{itemize}
} % new environment for highlights for bullet entries

\newenvironment{onecolentry}{
    \begin{adjustwidth}{
        0 cm + 0.00001 cm
    }{
        0 cm + 0.00001 cm
    }
}{
    \end{adjustwidth}
} % new environment for one column entries

\newenvironment{twocolentry}[2][]{
    \onecolentry
    \def\secondColumn{#2}
    \setcolumnwidth{\fill, 4.5 cm}
    \begin{paracol}{2}
}{
    \switchcolumn \raggedleft \secondColumn
    \end{paracol}
    \endonecolentry
} % new environment for two column entries

\newenvironment{threecolentry}[3][]{
    \onecolentry
    \def\thirdColumn{#3}
    \setcolumnwidth{, \fill, 4.5 cm}
    \begin{paracol}{3}
    {\raggedright #2} \switchcolumn
}{
    \switchcolumn \raggedleft \thirdColumn
    \end{paracol}
    \endonecolentry
} % new environment for three column entries

\newenvironment{header}{
    \setlength{\topsep}{0pt}\par\kern\topsep\centering\linespread{1.5}
}{
    \par\kern\topsep
} % new environment for the header

\newcommand{\placelastupdatedtext}{% \placetextbox{<horizontal pos>}{<vertical pos>}{<stuff>}
  \AddToShipoutPictureFG*{% Add <stuff> to current page foreground
    \put(
        \LenToUnit{\paperwidth-2 cm-0 cm+0.05cm},
        \LenToUnit{\paperheight-0.25 cm}
    ){\vtop{{\null}\makebox[0pt][c]{
        \small\color{gray}\textit{Last updated in December 2024}\hspace{\widthof{Last updated in December 2024}}
    }}}%
  }%
}%

% save the original href command in a new command:
\let\hrefWithoutArrow\href

% new command for external links:
\renewcommand{\href}[2]{\hrefWithoutArrow{#1}{\ifthenelse{\equal{#2}{}}{ }{#2 }\raisebox{.15ex}{\footnotesize \faExternalLink*}}}


\begin{document}
    \newcommand{\AND}{\unskip
        \cleaders\copy\ANDbox\hskip\wd\ANDbox
        \ignorespaces
    }
    \newsavebox\ANDbox
    \sbox\ANDbox{$|$}

    \begin{header}
        \fontsize{25 pt}{25 pt}\selectfont Kenny Garcia

        \vspace{5 pt}

        \normalsize
        \mbox{{\footnotesize\faMapMarker*}\hspace*{0.13cm}Whittier, CA}%
        \kern 5.0 pt%
        \AND%
        \kern 5.0 pt%
        \mbox{\hrefWithoutArrow{mailto:kennygarcia15@yahoo.com}{{\footnotesize\faEnvelope[regular]}\hspace*{0.13cm}kennygarcia15@yahoo.com}}%
        \kern 5.0 pt%
        \AND%
        \kern 5.0 pt%
        \mbox{\hrefWithoutArrow{tel:+1-562-536-2101}{{\footnotesize\faPhone*}\hspace*{0.13cm}(562) 536-2101}}%
        \kern 5.0 pt%
        \AND%
        \kern 5.0 pt%
        \mbox{\hrefWithoutArrow{https://bebopkenny.github.io/Portfolio/}{{\footnotesize\faLink}\hspace*{0.13cm}bebopkenny.github.io/Portfolio}}%
        \kern 5.0 pt%
        \AND%
        \kern 5.0 pt%
        \mbox{\hrefWithoutArrow{https://linkedin.com/in/linkedin.com/in/kennygarcia15}{{\footnotesize\faLinkedinIn}\hspace*{0.13cm}linkedin.com/in/kennygarcia15}}%
        \kern 5.0 pt%
        \AND%
        \kern 5.0 pt%
        \mbox{\hrefWithoutArrow{https://github.com/bebopkenny}{{\footnotesize\faGithub}\hspace*{0.13cm}bebopkenny}}%
    \end{header}

    \vspace{0.5 pt - 0.3 cm}


    \section{Education}



        
        \begin{twocolentry}{
            Sept 2023 – Dec 2026
        }
            \textbf{California State University Fullerton}, Bachelor of Science in Computer Science\end{twocolentry}

        \vspace{0.10 cm}
        \begin{onecolentry}
            \begin{highlights}
                \item \textbf{Coursework:} Data Structures, Algorithms, Operating Systems
            \end{highlights}
        \end{onecolentry}



    
    \section{Work Experience}



        
        \begin{twocolentry}{
            Nov 2024 – present
        }
            \textbf{Programming Instructor}, Coding Minds Academey -- Irvine, CA\end{twocolentry}

        \vspace{0.10 cm}
        \begin{onecolentry}
            \begin{highlights}
                \item Empowered K-12 students to develop programming skills through hands-on projects, leading to improved problem-solving and creativity
                \item Enhanced student engagement and understanding by teaching coding concepts in Python, JavaScript, and Scratch, increasing class satisfaction scores by 20\%, using interactive coding exercises and real-world applications
                \item Created a positive learning environment fostering student confidence and curiosity, leading to success in national coding competitions and Ivy League placements through personalized mentoring and curriculum adaptation
            \end{highlights}
        \end{onecolentry}



    
    \section{Projects}



        
        \begin{twocolentry}{
            Feb 2024
        }
            \textbf{CodeClimber} -- Fullerton, CA\end{twocolentry}

        \vspace{0.10 cm}
        \begin{onecolentry}
            \begin{highlights}
                \item Developed Code Climber, a web application for tracking coding interview progress on platforms like LeetCode and HackerRank, enabling users to save solutions, problem prompts, and complexity details
                \item Implemented the frontend using Next.js, HTML, and CSS, providing customizable user experiences and dynamic interface features to improve user engagement
                \item Built the backend using ExpressJS to manage data storage and server functionality, enabling users to access and update their progress across multiple devices
            \end{highlights}
        \end{onecolentry}


        \vspace{0.2 cm}

        \begin{twocolentry}{
            June 2023
        }
            \textbf{SympToDialog} -- Irvine, CA\end{twocolentry}

        \vspace{0.10 cm}
        \begin{onecolentry}
            \begin{highlights}
                \item Developed an AI-powered chatbot for healthcare provider training, enhancing diagnostic skills in sensitive STI conversations, using JavaScript, Node.js, and OpenAI API for realistic dialogue generation
                \item Built a scalable application architecture supporting real-time interaction and secure data management, leveraging Next.js and Firebase for seamless performance and user authentication
                \item Collaborated on UI design in Figma, creating an intuitive and visually appealing interface that received positive feed back
            \end{highlights}
        \end{onecolentry}



    
    \section{Leadership}



        
        \begin{twocolentry}{
            June 2024 – present
        }
            \textbf{Association for Computing Machinery}, AI Officer -- Fullerton, CA\end{twocolentry}

        \vspace{0.10 cm}
        \begin{onecolentry}
            \begin{highlights}
                \item Board member of the AI Team for ACM, actively organizing and leading workshops on algorithms, AI, and software development, with hands-on projects that simplify complex topics for students
                \item Organized and conducted Python workshops, focusing on AI libraries such as NumPy, Pandas, and PyTorch, enabling students to apply these tools in real-world scenarios through interactive code-along sessions
                \item Developed Google Colab worksheets to provide students with follow-along guides, enhancing their learning experience and offering a resource for future reference in projects and studies
            \end{highlights}
        \end{onecolentry}



    
    \section{Awards}



        
        \begin{twocolentry}{
            June 2023
        }
            \textbf{Winner of Youtube Health Innovation Prize}, National Academies Hackathon\end{twocolentry}

        \vspace{0.10 cm}
        \begin{onecolentry}
            \begin{highlights}
                \item Developed SympToDialog, an AI chatbot to train healthcare providers in STI diagnostics, improving confidence in taking sexual histories through diverse personas and AI technology using JavaScript, Node.js, and OpenAI API, earning the YouTube Health Innovation Prize solution using JavaScript, Node.js, and OpenAI API, earning the YouTube Health Innovation Prize
            \end{highlights}
        \end{onecolentry}



    
    \section{Skills}



        
        \begin{onecolentry}
            \textbf{Programming:} C++, Python, JavaScript, x86 Assembly
        \end{onecolentry}

        \vspace{0.2 cm}

        \begin{onecolentry}
            \textbf{Frameworks \& Libraries:} React.js, Node.js, Express.js, Flask, jQuery, EJS
        \end{onecolentry}

        \vspace{0.2 cm}

        \begin{onecolentry}
            \textbf{Tools:} Git, Unix, JSON, API, GUI, Matplotlib
        \end{onecolentry}

        \vspace{0.2 cm}

        \begin{onecolentry}
            \textbf{Databases:} SQL, PostgreSQL
        \end{onecolentry}


    

\end{document}